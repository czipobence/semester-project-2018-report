%%%%%%%%%%%%%%%%%%%%%%%%%%%%%%%%%%%%%%%%%
% Beamer Presentation
% LaTeX Template
% Version 1.0 (10/11/12)
%
% This template has been downloaded from:
% http://www.LaTeXTemplates.com
%
% License:
% CC BY-NC-SA 3.0 (http://creativecommons.org/licenses/by-nc-sa/3.0/)
%
%%%%%%%%%%%%%%%%%%%%%%%%%%%%%%%%%%%%%%%%%


%----------------------------------------------------------------------------------------
%	PACKAGES AND THEMES
%----------------------------------------------------------------------------------------

\documentclass{beamer}

\mode<presentation> {

% The Beamer class comes with a number of default slide themes
% which change the colors and layouts of slides. Below this is a list
% of all the themes, uncomment each in turn to see what they look like.

%\usetheme{default}
%\usetheme{AnnArbor}
%\usetheme{Antibes}
%\usetheme{Bergen}
%\usetheme{Berkeley}
%\usetheme{Berlin}
%\usetheme{Boadilla}
%\usetheme{CambridgeUS}
%\usetheme{Copenhagen}
%\usetheme{Darmstadt}
%\usetheme{Dresden}
%\usetheme{Frankfurt}
%\usetheme{Goettingen}
%\usetheme{Hannover}
%\usetheme{Ilmenau}
%\usetheme{JuanLesPins}
%\usetheme{Luebeck}
\usetheme{Madrid}
%\usetheme{Malmoe}
%\usetheme{Marburg}
%\usetheme{Montpellier}
%\usetheme{PaloAlto}
%\usetheme{Pittsburgh}
%\usetheme{Rochester}
%\usetheme{Singapore}
%\usetheme{Szeged}
%\usetheme{Warsaw}

% As well as themes, the Beamer class has a number of color themes
% for any slide theme. Uncomment each of these in turn to see how it
% changes the colors of your current slide theme.

%\usecolortheme{albatross}
%\usecolortheme{beaver}
%\usecolortheme{beetle}
%\usecolortheme{crane}
%\usecolortheme{dolphin}
%\usecolortheme{dove}
%\usecolortheme{fly}
%\usecolortheme{lily}
%\usecolortheme{orchid}
%\usecolortheme{rose}
%\usecolortheme{seagull}
%\usecolortheme{seahorse}
%\usecolortheme{whale}
%\usecolortheme{wolverine}

%\setbeamertemplate{footline} % To remove the footer line in all slides uncomment this line
%\setbeamertemplate{footline}[page number] % To replace the footer line in all slides with a simple slide count uncomment this line

%\setbeamertemplate{navigation symbols}{} % To remove the navigation symbols from the bottom of all slides uncomment this line
}

\usepackage{graphicx} % Allows including images
\usepackage{booktabs} % Allows the use of \toprule, \midrule and \bottomrule in tables

\usepackage[utf8]{inputenc}
\usepackage{soul}

%----------------------------------------------------------------------------------------
% RANDOM PACKAGES
%----------------------------------------------------------------------------------------


\usepackage[parfill]{parskip} %Skip initial spacing for each paragraph

%%Listings
\usepackage{algorithm}

\usepackage{enumerate}

%\usepackage[inline]{enumitem}
\usepackage{xspace}
\usepackage{listings}

\newtheorem{assumption}{Assumption}

\newcommand{\tf}{t}
\newcommand{\invtf}{t^{-1}}
\newcommand{\true}{\text{true}}
\newcommand{\false}{\text{false}}
\newcommand{\True}{\top}
\newcommand{\False}{\bottom}
\newcommand{\tr}[1]{[#1]}

\newcommand{\reduceTo}{\rightarrow^{*}}
\newcommand{\reduceToCoq}{\rightarrow^{*}_{\text{coq}}}

%% LISTINGS %%

\usepackage{xcolor}
\usepackage{lstcoq}

\definecolor{dkgreen}{rgb}{0,0.6,0}
\definecolor{gray}{rgb}{0.5,0.5,0.5}
\definecolor{mauve}{rgb}{0.58,0,0.82}
\definecolor{ltblue}{rgb}{0,0.4,0.4}
\definecolor{dkblue}{rgb}{0,0.1,0.6}
\definecolor{dkgreen}{rgb}{0,0.35,0}
\definecolor{dkviolet}{rgb}{0.3,0,0.5}
\definecolor{dkred}{rgb}{0.5,0,0}

\lstdefinestyle{myScalastyle}{
	frame=tb,
	language=scala,
	aboveskip=3mm,
	belowskip=3mm,
	showstringspaces=false,
	columns=flexible,
	basicstyle={\small\ttfamily},
	numbers=none,
	numberstyle=\tiny\color{gray},
	keywordstyle=\color{blue},
	commentstyle=\color{dkgreen},
	stringstyle=\color{mauve},
	frame=single,
	breaklines=true,
	breakatwhitespace=true,
	tabsize=3,
}

\lstdefinestyle{myCoqstyle}{
	frame=tb,
	language=coq,
	aboveskip=3mm,
	belowskip=3mm,
	showstringspaces=false,
	columns=flexible,
	basicstyle={\small\ttfamily},
	numbers=none,
	numberstyle=\tiny\color{gray},
	keywordstyle=\color{blue},
	commentstyle=\color{dkgreen},
	stringstyle=\color{mauve},
	frame=single,
	breaklines=true,
	breakatwhitespace=true,
	tabsize=3,
}

\newcommand{\scalaInline}[1]{\lstinline[style=myScalastyle]|#1|}
\newcommand{\coqInline}[1]{\lstinline[style=myCoqStyle]|#1|}
%----------------------------------------------------------------------------------------
%	TITLE PAGE
%----------------------------------------------------------------------------------------

\title[Semester project]{Verifying correctness of Stainless programs using Coq} % The short title appears at the bottom of every slide, the full title is only on the title page

\author{Bence Czipó 
	%\\{\small Advisors: Jad Hamza, Viktor Kunčak}
} % Your name
\institute[] % Your institution as it will appear on the bottom of every slide, may be shorthand to save space
{
École Polytechnique Fédérale de Lausanne \\ % Your institution for the title page
\medskip
\textit{bence.czipo@epfl.ch} % Your email address
}
\date{\today} % Date, can be changed to a custom date

\begin{document}

\begin{frame}
\titlepage % Print the title page as the first slide
\end{frame}


\begin{frame}
\frametitle{Overview} % Table of contents slide, comment this block out to remove it
\tableofcontents % Throughout your presentation, if you choose to use \section{} and \subsection{} commands, these will automatically be printed on this slide as an overview of your presentation
\end{frame}

%----------------------------------------------------------------------------------------
%	PRESENTATION SLIDES
%----------------------------------------------------------------------------------------

\section{Translation}

\begin{frame}
	\frametitle{The Translation Function}
	\begin{definition}[Translation function]
		Let $\tf$ be a function that assigns a (correctly typed) Coq program to every (correctly typed) Stainless program.
	\end{definition}

	\onslide<2->{ Though $\tf$ is not designed to be invertible, for simplicity let us introduce the $\invtf$ notation where $\invtf(\tf(p)) = p$.}
	
	\onslide<3->{Also for simplicity, let us denote $\tf(p)$ by $\tr{p}$.}
\end{frame}
\subsection{Basic Concepts}

\begin{frame}
	\frametitle{Uniqueness of names}
	
	Translation loses some scoping rules
	
	\onslide<2-> {Uniqueness of names is ensured by renaming.}
	
	\onslide<3->{$x \rightarrow x_0$,}
	\onslide<4->{\\ $x \rightarrow x_1$,\\...}
	
	\onslide<5->{
		Polymorphism is not supported, methods are needed to be renamed.
	}
	
	\onslide<6->{\scalaInline{def forall[T](l : List[T], p: T -> Boolean) -> forall0}}
	
	\onslide<7->{
		\scalaInline{def forall[T](o: Option[T], p: T -> Boolean) -> forall1}
		
		From now on, unique names are assumed.
	}
	
	
\end{frame}


\begin{frame}[fragile]{The Program Library}
	For example, we can have the following definition:
	\begin{lstlisting}[style=myCoqstyle]
	Definition add (n : nat)(m: nat) (p: n >= m) : { x : nat | x > 2 } :=
	Nat.add n 1.
	\end{lstlisting}
	
	The refined type will generate an obligation we have to prove:
	\begin{lstlisting}[style=myCoqstyle]
	$\forall$ n m : nat, n $\ge$ m $\rightarrow$ ($\lambda$ x : nat, x > 2) (n + 1)%nat
	\end{lstlisting}
\end{frame}

\begin{frame}[fragile]{The Program Library}
	Using add, we can give an example for obligations generated to plug in holes in types:
	
	\begin{lstlisting}[style=myCoqstyle]
	Definition mul2(n: nat): nat := add n n _.
	\end{lstlisting}
	
	It will generate require us to give an expression of the type of the missing part, specifically in this case
	
	\begin{lstlisting}[style=myCoqstyle]
	$\forall$ n: nat, n $\ge$ n
	\end{lstlisting}
\end{frame}


\begin{frame}
	\frametitle{Translating simple types}
	
	Most constructs have exact Coq representation
	
	\begin{itemize}
		\item $\tr{\text{BigInt}} = \tr{\text{Int}} = Z$
		
		\item $\tr{\text{Boolean}} =  bool$ (not Prop)
		
		\item $\tr{(a,b,\ldots)} = (\tr{a},\tr{b},\ldots)$
		
		\item $\tr{f(p_1,p_2,\ldots)} = (f ~ \tr{p_1} ~  \tr{p_2} \ldots)$
		
		\item $\tr{\lambda x. ~ e} = (\text{fun} ~ x ~ \Longrightarrow ~ \tr{e} \ldots)$
	\end{itemize}
	 
\end{frame}

\begin{frame}[fragile]{Translating error}
	Errors are translated to Coq using contradiction:
	
$\tr{\text{error}} = \text{contradiction} ~ \_$

Contradictions can be expressed as the obligation to prove false.
	
\begin{lstlisting}[style=myCoqstyle]
Definition contradiction (T: Type)(p: False) : T :=
	match p with 
	end.
\end{lstlisting}
\end{frame}

\begin{frame}[fragile]
	\frametitle{Dependent if-then-else}
	
	In the expression \scalaInline{if (p) then tb else fb}, we would like to propagate the boolean value of p to the branches.
	There are some solutions, but for more flexibility with the expressions, we defined our own.
	
	\begin{lstlisting}[style=myCoqstyle]
	Definition ifthenelse b A (e1: true = b -> A) (e2: false = b -> A): A :=
	match b as B return (B = b -> A) with
	| true => fun H => e1 H
	| false => fun H => e2 H
	end eq_refl.
	\end{lstlisting}
	Used for:
	\begin{itemize}
		\item if then else
		\item non-exhaustive matches
		\item boolean and (\scalaInline{\&\&})
	\end{itemize}
	
	
\end{frame}

\begin{frame}[fragile]
	\frametitle{Translating equality}
	Equality is usually translated using the coq equality.
	
		In Coq, \coqInline{a = b} is of type \coqInline{Prop}.
	
\begin{lstlisting}[style=myCoqstyle]
Definition propInBool (P: Prop): bool :=
if (classicT P)
then true
else false.
\end{lstlisting}

Decidable type system: every Prop is True or False

\begin{lstlisting}[style=myCoqstyle]
Axiom classicT: forall P: Prop, P + ~P.
\end{lstlisting}

\onslide<2-> {
	Exceptions: BigInt, Int, Boolean and Sets
}

\end{frame}

\begin{frame}{Translating Sets}
	The Coq standard library comes with a limited set implementation.
	
	\onslide<2->{Coq-std++ (stdpp): collection of data structures, lemmas and tactics.}
	
	\onslide<3->{Contains sets with every common operation and \coqInline{set_solver} tactic to solve obligations }
\end{frame}

\begin{frame}{The Program Library}
	Program is a Coq library that:
	\begin{itemize}
		\item allows us to convert between types and dependent types implicitly
		\onslide<2->{\item generates obligations to "plug in" holes in the context}
		\onslide<3->{\item also allows incomplete missing parameters, for which, it will generate obligations}
		\onslide<4->{\item defines an Obligation Tactic to solve the generated obligations
		}
	\end{itemize}
	
\end{frame}

\begin{frame}
	\frametitle{Structure}
	Stainless programs from our point of view are:
	\onslide<2-> {
		\begin{itemize}
			\item ADT Definitions
			\onslide<3-> {
				\begin{itemize}
					\item ADTSort
					\onslide<4->{\item ADTConstructor}
				\end{itemize}
			}
			\onslide<5-> {\item Methods on them}
				\onslide<6-> {
				\begin{itemize}
					\item Non-recursive
					\onslide<7->{\item Recursive}
					\only<8>{\item Mutually Recursive}
					\onslide<9>{\item \st{Mutually Recursive}}
				\end{itemize}
			}
		\end{itemize}
	}
	
\end{frame}

\subsection{Transforming Abstract Data Types}

\begin{frame}
	\Huge{\centerline{Transforming Abstract Data Types}}
\end{frame}

\begin{frame}[fragile]
	\frametitle{List Example}
	
	Best explained through an example:
	
	\begin{lstlisting}[style=myScalastyle]  
	sealed abstract class List[T] {}
	
	case class Nil[T]() extends List[T] {}
	
	case class Cons[T](h: T, t: List[T]) extends List[T] {}
	\end{lstlisting}
	
\end{frame}

\begin{frame}[fragile]
	\frametitle{Type definitions}
	
	Inductive type definitions for ADTSorts. The sematntics behind this is that an ADTSort is one of its constructors.
	
	\begin{lstlisting}[style=myCoqstyle]  
	Inductive List (T: Type) :=
	| Cons: T -> ((List T) -> (List T))
	| Nil: List T.
	\end{lstlisting}
	
\end{frame}

\begin{frame}[fragile]
	\frametitle{Recognizing Types}
	
	We can use pattern matching to check for concrete subtype
	
\begin{lstlisting}[style=myCoqstyle]  
Definition isCons (T: Type) (src: List T) : bool :=
match src with
| Cons _ _ _ => true
| _ => false
end.
\end{lstlisting}

Using the recognizers, we can define a type for the subtypes as a refined type.

\begin{lstlisting}[style=myCoqstyle]  
Definition Cons_type (T: Type) : Type :=
{src: List T | (isCons T src = true)}. 
\end{lstlisting}
	
\end{frame}

\begin{frame}[fragile]
\frametitle{Accessing Fields}

Now that we have the subtypes, we can have accessors to their fields

\begin{lstlisting}[style=myCoqstyle]  
Definition h (T: Type) (src: Cons_type T) : T :=
match src with
| Cons_construct _ f0 f1 => f0
| _ => let contradiction: False := _ in match contradiction with end
end. 
\end{lstlisting}

\end{frame}


\subsection{Transforming Methods}

\begin{frame}[fragile]
\frametitle{Methods}
A general method is built up from
\begin{itemize}
	\item a function name $f$
	\item type parameters $T_1 \ldots T_k$ and arguments $p_1$ of type $U_1$, $p_2$ of type $U_2$ ... , $p_n$ of type $U_n$ and a return type $U_r$
	\item a precondition pre of type \scalaInline{A => Boolean}, where $A \preceq \{U_1 x \ldots x U_n\}$
	\item a body $b$ of type $\{U_1 x \ldots x U_n\} \Longrightarrow U_r$
	\item a postcondition $post$ of type $U_r$\scalaInline{ => Boolean } 
\end{itemize}

If there are more pre- and postcondition, they can be combined into one pre- and postcondition using conjunction.

\end{frame}

\begin{frame}
	\Huge{\centerline{Transforming Methods}}
\end{frame}

\begin{frame}
	\frametitle{Transforming non-recursive methods}
	
	If there is no recursion involved, the translation is straightforward.
	
	\onslide<2->{Preconditions can be expressed in Coq by taking an argument that states that the precondition holds, in oder means, taking an argument with the type $\text{pre} = \true$.}
	
	
	\onslide<3->{Postconditions can be expressed by a dependent return type. In case of a postcondition  $\lambda  \text{res}. ~\text{post}( \text{res})$ the return type changes to $\{\text{res}: U_r ~|~ \text{post}( \text{res})\}$. 
	}	
\end{frame}

\begin{frame}[fragile]
\frametitle{An example}
\begin{lstlisting}[style=myScalastyle]
def f[T1 ... Tk](p1: U1, ... pn: Un): Ur = {
require(pre(A))
b
} ensuring {res => post(res)}
\end{lstlisting}

Is translated to

\begin{lstlisting}[style=myCoqstyle]  
Definition f ($T_1$: Type) ... ($T_k$: Type)($p_1$: $\tr{U_1}$) $\ldots$ ($p_n$: $\tr{U_n}$) (prec: $\tr{\text{pre}}$ = true) : 
{res: $\tr{U_r} ~|~ \tr{\text{post}}$ res} :=
$\tr{b}$.
\end{lstlisting}

\end{frame}


\subsection{Transforming Recursive Methods}


\begin{frame}
	\Huge{\centerline{Transforming Recursive Methods}}
\end{frame}


\begin{frame}[fragile]
	\frametitle{Translating Recursive Methods}
	Program library has Program Fixpoint to write recursive functions.
	
	Makes it extremely hard to rewrite.
	
	We used \emph{CoqEquations} instead.
	
\begin{lstlisting}[style=myCoqStyle]
Equations negb (b : bool) : bool :=
	negb true := false ;
	negb false := true.
\end{lstlisting}
\end{frame}


\begin{frame}[fragile]
	\frametitle{Persevering benefits of Program}
	We still want to handle pre- and postconditions using Program.
	
	We can translate preconditions into dependent types:
	\begin{lstlisting}[style=myCoqstyle]  
Definition prt ($T_1$: Type) ... ($T_k$: Type)($p_1$: $\tr{U_1}$) $\ldots$ ($p_n$: $\tr{U_n}$) :
Type := $\tr{\text{pre}}$ = true
	\end{lstlisting}
	
	And postconditions just the same:
	\begin{lstlisting}[style=myCoqstyle] 
Definition rt ($T_1$: Type) ... ($T_k$: Type)($p_1$: $\tr{U_1}$) $\ldots$ ($p_n$: $\tr{U_n}$) (prec: prt $T_1$ $\ldots$ $T_k$ $p_1$ $\ldots$ $p_n$) : Type :=
{res: $\tr{U_r} ~|~ \tr{\text{post}}$ res}
	
	\end{lstlisting}
\end{frame}

\begin{frame}[fragile]
	\frametitle{Persevering benefits of Program}
	The previous example in the recursive case would be translated to:
	
	\begin{lstlisting}[style=myCoqstyle]
Equations f ($T_1$: Type) ... ($T_k$: Type)
			     ($p_1$: $\tr{U_1}$) $\ldots$ ($p_n$: $\tr{U_n}$) 
			     (prec: prt $T_1$ $\ldots$ $T_k$ $p_1$ $\ldots$ $p_n$) : 
			     rt $T_1$ $\ldots$ $T_k$ $p_1$ $\ldots$ $p_n$ prec :=
						f $T_1$ $\ldots$ $T_k$ $p_1$ $\ldots$ $p_n$ prec by rec ignore_termination lt :=
						f $T_1$ $\ldots$ $T_k$ $p_1$ $\ldots$ $p_n$ prec :=
								$\tr{b}$.
	\end{lstlisting}
	
	The \coqInline{ignore\_termination} is the decreasing measure in the recursion. We will have an obligation proving it to be decreasing.
	
	Later, if we want to rewrite with the content of the function, we can rely on the fact that it will be expressed by \coqInline{f\_equation\_1}.
	
\end{frame}


\begin{frame}[fragile]
	\frametitle{Function Application}
	
	How to write function application depends on whether the it has pre- or postcondition
	
	\begin{itemize}
		\item If the method had preconditions, we pass \_, so that Program will generate obligations for us.
		\item If there is a postcondition, the result has to be projected
	\end{itemize}

\begin{lstlisting}[style=myCoqstyle]  
proj1_sig (f T1 ... Tj p1 ... pn _)
\end{lstlisting}

\end{frame}

\section{Relation Between Proofs and Correctness}
\begin{frame}
	\Huge{
		\begin{center}
			Relation Between Proofs and Correctness
		\end{center}
	}
\end{frame}

\begin{frame}{Validity - Stainless}
	\begin{definition}[Stainless-validity]
		A Stainless program $p$ is valid if
		
		\begin{itemize}
			\item the preconditions hold for every call of $f$
			\item for any input satisfying the preconditions, the postcondition holds
		\end{itemize}
	\end{definition}

	\onslide<2->{
		Pre- and postcondition branching, failing postcondition go into an error branch.
	Errors are translated to Coq using contradiction: $$\tr{\text{error}} = \text{contradiction} ~ \_$$
	}
	\onslide<3->{
	Applying \coqInline{contradiction} on an unknown variable will result in an obligation to derive \coqInline{False} from the context.}
	
\end{frame}

\begin{frame}{Validity - Coq}
 \begin{definition}[Coq-validity]
 	A Coq program $p$ is valid if there is a proof (an expression of that type) for every obligation generated by it.
 \end{definition}

\onslide<2->{We can define relation between the two}

\onslide<3->{\begin{theorem}
		For every (correctly typed) Stainless program $p$ and its translation $\tr{p}$, if $\tr{p}$ is proved to be valid by Coq, than $p$ is also valid.
\end{theorem}}

\onslide<4->{Instead of proving this we will just sketch some notions why is it true.
}

\end{frame}

\begin{frame}{Proof(ish)}
	
	\begin{assumption}
		\label{ass:bools}
		For every every Scala term \scalaInline{b: Boolean} we assume that if $b \reduceTo c$ where $c \in \{\text{true}, \text{false}\}$ then $\tr{b} \reduceToCoq \tr{c}$.
	\end{assumption}

\onslide<2->{
If b evaluates to a boolean constant in Stainless (under the context $\Gamma$), its translated representation evaluate to the representation of that boolean constant in Coq (under the context $\tr{\Gamma}$)
}

\onslide<3-> {
\begin{assumption}
	Let us assume that in the program $p$, every decision is a branching based on a boolean condition.
\end{assumption}
}

\onslide<4-> {
	\begin{assumption}
		Let us assume that $p$ does not contain any free variables.
	\end{assumption}
}

\end{frame}

\begin{frame}{Proof(ish)}
	By contradiction: assume $\tr{p}$ is proven to be correct, but $p$ crashes
	
	\onslide<2->{
		there is a branch in p that contains error, and it is accessed through evaluating the boolean expressions $b_1, b_2 \ldots b_n$
	}

	\onslide<3->{$\tr{p}$ reduces to the same branch $\rightarrow$ contradiction}
	
	\onslide<4- >{Coq was able to prove it: $\{\} \vdash x: False$}
	
	\onslide<5->{
	
	\begin{fact}
		In Coq, if $\{\} \vdash t: T$, then $t \reduceToCoq v$, where v is a value of $T$
	\end{fact}
	\begin{fact}
		There is no value of type \coqInline{False}.
	\end{fact}
	}	
\end{frame}

\section{Automated Verification}

\begin{frame}
	\Huge{\centerline{Automated Verification}}
\end{frame}


\begin{frame}
	\frametitle{Automatic Verification}
	
	How to solve obligations automatically?
	
	\onslide<2->{Define obligation tactic}
	
	\begin{enumerate}
		\onslide<3->{\item Fast tactics}
		\onslide<3->{\item Basic Tactics}
		\onslide<3->{\item Slow Tactics}
		\onslide<3->{\item Set Tactics}
		\onslide<3->{\item Case Analysis}
		\onslide<3->{\item Rewrite}
		
	\end{enumerate}
	
\end{frame}

\begin{frame}
	\frametitle{Fast tactics}
	
	Some Coq tactics are fast to fail:
	\begin{itemize}
		\item cbn
		\item intros
		\item intuition
		\item discriminate
		\item ...
	\end{itemize}
\end{frame}



\begin{frame}[fragile]
	\frametitle{Fast Rewrites}
	Integer operations:
	
	\coqInline{
		forall x y : Z, (x <=? y) = false <->  y < x}
	
	Boolean operations:
	\coqInline{forall a b : bool, eqb a b = true <-> a = b}

	\onslide<2-> {
		Own rewrite lemmas about booleans: \\
		\coqInline{  forall b1 b2,
			negb b1 = negb b2 <-> b1 = b2}
	}
	
	\onslide<3-> {
		Rewrite lemmas with Props and bools:
		\coqInline{forall P, propInBool P = true <-> P}
	}
	
	\onslide<4->{
		Own rewrite lemmas about if-then-else: \\		
		\coqInline{forall b: bool,  (if b then true else false) = b.}
	}

\end{frame}

\begin{frame}[fragile]
	\frametitle{Fast Rewrites}
	
	
	Also some rewrite lemmas with dependent if-then-else:
\begin{lstlisting}[style=myCoqstyle]
 forall T b e1 e2 value,
	ifthenelse b T e1 e2 = value <-> (
	(exists H1: true = b, e1 H1 = value) \/
	(exists H2: false = b, e2 H2 = value)
	).
\end{lstlisting}
	
	or
	
\begin{lstlisting}[style=myCoqstyle]
forall b (e1: true = b -> bool),
	ifthenelse b bool e1 (fun _ => false) = true <->
	exists H: true = b, e1 H = true.
\end{lstlisting}
	
\end{frame}

\begin{frame}[fragile]
	\frametitle{Admitting termination obligations}
	
	Some obligations are related to termination, that we ignore currently.
	
	We will have \coqInline{(ignore\_termination < ignore\_termination)\%nat} in the goal
	
	Specific tactic to recognize it, and admit it.
	
	\begin{lstlisting}[style=myCoqstyle]
match goal with
(...)
| |- (S ?T <= ?T)%nat =>
	unify T ignore_termination; apply False_ind; exact unsupported
(...)
end.
	\end{lstlisting}
	
\end{frame}


\begin{frame}
	\frametitle{Basic Tactics}
	\begin{itemize}
		\item Rewriting with boolean constants and value
		\item Destruction of exists, refinement, etc...
		\item Prop $\leftrightarrow$ bool rewrites
	\end{itemize}
\end{frame}


\begin{frame}{Slow Tactics}
	Coq's "not-so-fast" tactics:
	\begin{itemize}
		\item omega
		\item ring
		\item eauto (currently not included)
	\end{itemize}
\end{frame}

\begin{frame}{Set Tactics}
	Solve sets using \coqInline{set\_solver} of stdpp. 
	
	Also includes some basic set rewrites before:
	\begin{itemize}
		\item $\emptyset \cup s = s$
		\item $s \cup \emptyset = s$
		\item $s_1 = s_2 \implies s_3 \cup s_1 = s_3 \cup s_2  $
		\item $s_1 = s_2 \leftrightarrow s_1 \subseteq s_2 \wedge s_2 \subseteq s_1$
		\item $x \in s_1 \vee x \in s_2 \leftrightarrow x \in (s_1 \cup s_2)$
		\item ...
	\end{itemize}	
\end{frame}

\begin{frame}
	\frametitle{Case Analysis}
	If we encounter an if-then-else (both dependent and non-dependent) or a match we can 
	\begin{itemize}
		\item rewrite using some smart rule, e.g.
		
		\onslide<2->{
			\coqInline{(forall H1: true = b, e1 H1 = value) ->}
			\coqInline{
				(forall H2: false = b, e2 H2 = value) ->}
			\coqInline{
				ifthenelse b T e1 e2 = value. }}
		\onslide<3->{\item perform case-analysis}
	\end{itemize}
	
	
\end{frame}

\begin{frame}
	\frametitle{Rewrites}
	Rewrites can cause loops and obfuscate the goal. 
	
	The order is important:
	\onslide<2->{
		\begin{enumerate}
			\item Rewrite with non-recursive types
			\onslide<3->{\item Rewrite with the body of recursive functions}
			\onslide<4->{\item Rewrite with recognizers (\coqInline{isXY})}
		\end{enumerate}
	}
	\onslide<5->{Rewriting with the body of recursive methods does not happen here, rather, they are just put into the context as equations.}
	
	\onslide<6->{Right after fast tactics, whenever we see an \textbf{appropriate} equation we rewrite with it}
\end{frame}

\begin{frame}[fragile]
	\frametitle{Appropriate?}
	We only rewrite with the body of recursive functions, if their body will probably not be the subject of further refinement. 
	
	\begin{lstlisting}[style=myCoqstyle]
	Rw: size T l = match l with
	 | Nil T => 0
	 | Cons T x xs => 1 + size T xs
	end.
	\end{lstlisting}
	This will just introduce more branches to deal with, and we would like to perform destructing before rewriting. 
	
	
	
\end{frame}

\begin{frame}[fragile]
	\frametitle{Appropriate?}
	However, if we know that \coqInline{l} is cons:
\begin{lstlisting}[style=myCoqstyle]
l, ys : List T
y: T
H : l = Cons y ys
(...)
Rw : size T l = match (Cons y ys) with
| Nil T => 0
| Cons T x xs => 1 + size T xs
end.
\end{lstlisting}

	We can simplify it to: 
	
\begin{lstlisting}[style=myCoqstyle]
	Rw: size T l = 1 + size T ys
\end{lstlisting}
\end{frame}

\section{Implementation}

\begin{frame}
	\Huge{\centerline{Implementation}}
\end{frame}

\begin{frame}
	\frametitle{Implementation}
	
	Integrate into Stainless toolchain
	
	\onslide<2->{Generate separate files per function, that includes all dependencies}
	
	\onslide<3->{
		\begin{itemize}
			\item All ADT's
			\item (Transitively) invoked functions
		\end{itemize}
	}

	\onslide<4-> {
		\coqInline{Admit Obligations} for dependencies
		
		\begin{itemize}
			\item Saves time
			\item Eliminates domino effect
		\end{itemize}
	
		A method is valid if the verification condition generated for it is valid, and the verification conditions of all of its dependencies are correct too.
	}
\end{frame}

\begin{frame}
	\frametitle{Architecture}
	VerificationChecker $\rightarrow$ CoqVerificationChecker
	
	\onslide<2-> {CoqIO.scala: invoke coqc and check the output}

	\onslide<3-> {
		Possible output:
		\begin{itemize}
			\item Valid
			\onslide<4->{\item Invalid: the execution terminated without solving one or more obligations}
			\onslide<5->{\item Timeout: the verification timed out}
			\onslide<6->{\item Error: the verification failed, because of some internal error, most likely an error in the generated file}
			\onslide<7->{\item Canceled: the verification was canceled}
		\end{itemize}
	}
	
\end{frame}


\section{Benchmark}

\begin{frame}
	\Huge{\centerline{Benchmark}}
\end{frame}

\begin{frame}
	\frametitle{List library}
	\begin{itemize}
		\item Recursive, but not mutually recursive methods
		\item Inductive ADT's
		\item Set operations (contains, content, intersection, ...)
		\item Integer operations (size, indexOf, indexWhere, ...)
		
		
		\onslide<2->{\item 77 methods in total}		
	\end{itemize}

\end{frame}

\begin{frame}
	\frametitle{Results}
	
	Verification run with 5 minutes timeout
	
	\begin{itemize}
		\item Valid: 66 
		\item Invalid: 0
		\item Timeout: 11
		\item Error: 0 
		
	\end{itemize}
\end{frame}

\begin{frame}
	\frametitle{Conclusions}
	\begin{itemize}
		\item Translating Stainless programs to Coq
		\item Tactics to solve generated goals
		\item Benchmark using List library  (66/77 verified)
	\end{itemize}
	
	\onslide<2-> {
		Future work: 
		\begin{itemize}
			\item Extending the supported stainless expressions
			\item Enhance tactics so that they handle the whole library
			\item Speed up tactics, external \coqInline{set-solver} tactic is really slow to fail, blocks every execution requiring a rewrite with definition
			\item Check termination
		\end{itemize}
		
	}
\end{frame}



\begin{frame}
	\Huge{\centerline{Questions?}}
\end{frame}


\end{document} 